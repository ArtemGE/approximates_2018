\documentclass[a4paper,12pt]{article}
\usepackage{graphicx}
\usepackage{bm,amssymb}
\usepackage{mathrsfs}
\usepackage[unicode,colorlinks=true,filecolor=blue, menucolor=black, linkcolor=black, citecolor=black,pagebackref=white]{hyperref}
\usepackage[utf8]{inputenc}
\usepackage[russian]{babel}
\usepackage{amsmath}
\usepackage{caption}
\usepackage[left=2cm,right=2cm, top=2cm,bottom=2cm,bindingoffset=0cm]{geometry}
\begin{document}
\textbf{1.} Теория возмущений, задача 1. Считая параметр $a>b$ $\varepsilon\ll1$, исследуем с помощью теории возмущений матрицу 
\[
\hat{H}=
\begin{pmatrix}
a & \varepsilon \\
\varepsilon & b 
\end{pmatrix}
\]

Найдите с помощью как невырожденной, так и вырожденной теории возмущений первую неисчезающую поправку к собственным числам. Какое условие применимости и того, и другого способа? Сравните с точным ответом.

\textbf{Ответ}
$$
\lambda_1^{(0)}=a
$$

$$
\lambda_2^{(0)}=b
$$
Из невырожденной теории возмущений получаем:
$$
\lambda_{1,2}^{(1)}=\pm\frac{\varepsilon^2}{a-b}
$$

$$
\lambda_{1,2}^{(1)}=\pm\varepsilon
$$

Точный ответ:
$$
\lambda_{1,2}=\frac{a+b\pm\sqrt{(a-b)^2+4\varepsilon^2}}{2}
$$

В пределе $\varepsilon\ll\left(a-b\right)$ получаем ответ невырожденной теории возмущений, а в обратном случае $\varepsilon\gg\left(a-b\right)$ - вырожденной. В случае $\varepsilon\sim\left(a-b\right)$, даже если $\varepsilon\ll1$, применять теорию возмущений нельзя!

\textbf{2.} Семинар по трансцендентным уравнениям. \textbf{Задача-шутка}. Метод итеративного решения уравнения задачи 3 на семинаре нельзя продолжать буквально до бесконечности, т.к. при некотором $N$:
$$
\ln\ln... A<0
$$
Число логарифмов в последнем неравенстве равно $N$. Оцените $N$ при $A\gg1$. 

\textbf{Решение}

Отношение наблюдаемого диаметра Вселенной к планковскому масштабу $\sim 10^{90}\approx e^{207}$. В то же время, $e\approx 2.7$, $e^e\approx15.2$, $e^{e^e}\approx3.8\cdot10^6$, и $e^{e^{e^e}}\gg10^{90}$ (здесь <<башню>> из экспонент надо читать сверху вниз), поэтому при любом $A$, которое встретится в реальной физической задаче, $N\sim1$. Т.о., при $A>3.8\cdot10^6$ $N=5$, при $A>15.2$ $N=4$, при $A>2.7$ $N=3$.

\textbf{3.} Метод перевала. Оценить интеграл
$$
\int_{1}^{10}x^x dx
$$

\textbf{Решение} Интеграл набирается вблизи точки $x=10$. Имеем:
$$
\int_{1}^{10}x^x dx\approx10^{10}\int_1^{10}dx e^{-(10-x)(1+\ln10)}\approx\frac{10^{10}}{1+\ln10}\approx3.03\cdot10^9
$$
Численный счёт даёт ответ $3.06\cdot10^9$. Точность порядка 1 процента.

\textbf{4.} Криволинейные интегралы. Вычислить среднее от функции $\frac{1}{r}$ по сфере радиуса $R$ и центром в точке $(X,Y,Z)=r_0$.

\textbf{Решение} 
\begin{eqnarray}
\langle\frac{1}{r}\rangle=\frac{1}{4\pi R^2}\int_{|r-r_0|=R}d^2r\frac{1}{r}=\frac{1}{4\pi R^2}\int_{|r|=R}d^2r\frac{1}{|r+r_0|}=\frac{1}{2}\int_{-1}^{1}\frac{dx}{\sqrt{R^2+r_0^2+2Rr_0 x}}={}
	\nonumber\\
{}=\frac{1}{Rr_0}\left(R+r_0-|R-r_0|\right)=\begin{cases} \frac{2}{R}, & R>r_0\\
\frac{2}{r_0}, & R<r_0 \end{cases}
\end{eqnarray}

\textbf{5.} Преобразование Фурье. 

\textbf{a.} Используя формулу для сопротивления между произвольными узлами квадратной решётки, полученную на лекции, найдите асимптотику $R(N_x, N_y)$ при $N\to\infty$, где $N=\sqrt{N_x^2+N_y^2}$.

\textbf{Решение}
\begin{eqnarray}
R_{N}=R\int\frac{\sin^2\left(\frac{\bm q \bm N}{2}\right)}{1-\frac{1}{2}\left(\cos q_x+\cos q_y\right)}\frac{d^2 \bm q}{(2\pi)^2}=\int_{q<N^{-1}}+\int_{q>N^{-1}}\approx{}
									\nonumber\\
{}\approx R\int_{q<N^{-1}}\frac{\bm q^2 \bm N^2}{q^2}\frac{d^2\bm q}{(2\pi)^2}+\frac{R}{2}\int_{q>N^{-1}}4\frac{d^2 \bm q}{(2\pi)^2 q^2}\approx\frac{R}{\pi}\ln N
\end{eqnarray}

\textbf{b.} Воспроизведите результат предыдущего пункта в непрерывном пределе

\textbf{c.} Пусть $N_x=N_y=N$. Вычислите $R(N,N)$

\textbf{*d.} Итак, нами было получено, что при $N\to\infty$ $R_N\to\infty$. Рассмотрим теперь одномерную и трёхмерную сетку сопротивлений. Как ведёт себя $R_{N\gg 1}$ в таком случае? Можно ли связать между собой $R_N$ и вероятность невозврата? 
\end{document}
